\documentclass[11pt]{article}
\usepackage[utf8]{inputenc}
\usepackage[T1]{fontenc}
\usepackage[english]{babel}
\usepackage[a4paper,margin=1in]{geometry}
\usepackage{amsmath,amssymb}
\usepackage{hyperref}

\title{Literature Review — Test Document}
\author{Ryan Habis}
\date{\today}

\begin{document}
\maketitle

\begin{abstract}
This short document verifies that Overleaf (or an Overleaf extension) compiles and renders a typical LaTeX file. It includes text, lists, math, and a simple bibliography.
\end{abstract}

\section{Introduction}
This is a minimal LaTeX example placed in the file to confirm the editor/compiler integration is functioning.

\section{Content Examples}
Inline math example: $a^2 + b^2 = c^2$.  

Displayed equation:
\[
\int_0^1 x^2 \, dx = \tfrac{1}{3}.
\]

\subsection{Bullet List}
\begin{itemize}
    \item First test item
    \item Second test item
    \item Third test item
\end{itemize}

\section{Conclusion}
If this file compiles and displays correctly in Overleaf, the extension is working.

\section*{References}
\begin{thebibliography}{9}
\bibitem{example}
A. Author, \textit{An Example Reference}, 2020.
\end{thebibliography}

\end{document}